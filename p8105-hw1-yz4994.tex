% Options for packages loaded elsewhere
\PassOptionsToPackage{unicode}{hyperref}
\PassOptionsToPackage{hyphens}{url}
%
\documentclass[
]{article}
\usepackage{amsmath,amssymb}
\usepackage{iftex}
\ifPDFTeX
  \usepackage[T1]{fontenc}
  \usepackage[utf8]{inputenc}
  \usepackage{textcomp} % provide euro and other symbols
\else % if luatex or xetex
  \usepackage{unicode-math} % this also loads fontspec
  \defaultfontfeatures{Scale=MatchLowercase}
  \defaultfontfeatures[\rmfamily]{Ligatures=TeX,Scale=1}
\fi
\usepackage{lmodern}
\ifPDFTeX\else
  % xetex/luatex font selection
\fi
% Use upquote if available, for straight quotes in verbatim environments
\IfFileExists{upquote.sty}{\usepackage{upquote}}{}
\IfFileExists{microtype.sty}{% use microtype if available
  \usepackage[]{microtype}
  \UseMicrotypeSet[protrusion]{basicmath} % disable protrusion for tt fonts
}{}
\makeatletter
\@ifundefined{KOMAClassName}{% if non-KOMA class
  \IfFileExists{parskip.sty}{%
    \usepackage{parskip}
  }{% else
    \setlength{\parindent}{0pt}
    \setlength{\parskip}{6pt plus 2pt minus 1pt}}
}{% if KOMA class
  \KOMAoptions{parskip=half}}
\makeatother
\usepackage{xcolor}
\usepackage[margin=1in]{geometry}
\usepackage{color}
\usepackage{fancyvrb}
\newcommand{\VerbBar}{|}
\newcommand{\VERB}{\Verb[commandchars=\\\{\}]}
\DefineVerbatimEnvironment{Highlighting}{Verbatim}{commandchars=\\\{\}}
% Add ',fontsize=\small' for more characters per line
\usepackage{framed}
\definecolor{shadecolor}{RGB}{248,248,248}
\newenvironment{Shaded}{\begin{snugshade}}{\end{snugshade}}
\newcommand{\AlertTok}[1]{\textcolor[rgb]{0.94,0.16,0.16}{#1}}
\newcommand{\AnnotationTok}[1]{\textcolor[rgb]{0.56,0.35,0.01}{\textbf{\textit{#1}}}}
\newcommand{\AttributeTok}[1]{\textcolor[rgb]{0.13,0.29,0.53}{#1}}
\newcommand{\BaseNTok}[1]{\textcolor[rgb]{0.00,0.00,0.81}{#1}}
\newcommand{\BuiltInTok}[1]{#1}
\newcommand{\CharTok}[1]{\textcolor[rgb]{0.31,0.60,0.02}{#1}}
\newcommand{\CommentTok}[1]{\textcolor[rgb]{0.56,0.35,0.01}{\textit{#1}}}
\newcommand{\CommentVarTok}[1]{\textcolor[rgb]{0.56,0.35,0.01}{\textbf{\textit{#1}}}}
\newcommand{\ConstantTok}[1]{\textcolor[rgb]{0.56,0.35,0.01}{#1}}
\newcommand{\ControlFlowTok}[1]{\textcolor[rgb]{0.13,0.29,0.53}{\textbf{#1}}}
\newcommand{\DataTypeTok}[1]{\textcolor[rgb]{0.13,0.29,0.53}{#1}}
\newcommand{\DecValTok}[1]{\textcolor[rgb]{0.00,0.00,0.81}{#1}}
\newcommand{\DocumentationTok}[1]{\textcolor[rgb]{0.56,0.35,0.01}{\textbf{\textit{#1}}}}
\newcommand{\ErrorTok}[1]{\textcolor[rgb]{0.64,0.00,0.00}{\textbf{#1}}}
\newcommand{\ExtensionTok}[1]{#1}
\newcommand{\FloatTok}[1]{\textcolor[rgb]{0.00,0.00,0.81}{#1}}
\newcommand{\FunctionTok}[1]{\textcolor[rgb]{0.13,0.29,0.53}{\textbf{#1}}}
\newcommand{\ImportTok}[1]{#1}
\newcommand{\InformationTok}[1]{\textcolor[rgb]{0.56,0.35,0.01}{\textbf{\textit{#1}}}}
\newcommand{\KeywordTok}[1]{\textcolor[rgb]{0.13,0.29,0.53}{\textbf{#1}}}
\newcommand{\NormalTok}[1]{#1}
\newcommand{\OperatorTok}[1]{\textcolor[rgb]{0.81,0.36,0.00}{\textbf{#1}}}
\newcommand{\OtherTok}[1]{\textcolor[rgb]{0.56,0.35,0.01}{#1}}
\newcommand{\PreprocessorTok}[1]{\textcolor[rgb]{0.56,0.35,0.01}{\textit{#1}}}
\newcommand{\RegionMarkerTok}[1]{#1}
\newcommand{\SpecialCharTok}[1]{\textcolor[rgb]{0.81,0.36,0.00}{\textbf{#1}}}
\newcommand{\SpecialStringTok}[1]{\textcolor[rgb]{0.31,0.60,0.02}{#1}}
\newcommand{\StringTok}[1]{\textcolor[rgb]{0.31,0.60,0.02}{#1}}
\newcommand{\VariableTok}[1]{\textcolor[rgb]{0.00,0.00,0.00}{#1}}
\newcommand{\VerbatimStringTok}[1]{\textcolor[rgb]{0.31,0.60,0.02}{#1}}
\newcommand{\WarningTok}[1]{\textcolor[rgb]{0.56,0.35,0.01}{\textbf{\textit{#1}}}}
\usepackage{graphicx}
\makeatletter
\newsavebox\pandoc@box
\newcommand*\pandocbounded[1]{% scales image to fit in text height/width
  \sbox\pandoc@box{#1}%
  \Gscale@div\@tempa{\textheight}{\dimexpr\ht\pandoc@box+\dp\pandoc@box\relax}%
  \Gscale@div\@tempb{\linewidth}{\wd\pandoc@box}%
  \ifdim\@tempb\p@<\@tempa\p@\let\@tempa\@tempb\fi% select the smaller of both
  \ifdim\@tempa\p@<\p@\scalebox{\@tempa}{\usebox\pandoc@box}%
  \else\usebox{\pandoc@box}%
  \fi%
}
% Set default figure placement to htbp
\def\fps@figure{htbp}
\makeatother
\setlength{\emergencystretch}{3em} % prevent overfull lines
\providecommand{\tightlist}{%
  \setlength{\itemsep}{0pt}\setlength{\parskip}{0pt}}
\setcounter{secnumdepth}{-\maxdimen} % remove section numbering
\usepackage{bookmark}
\IfFileExists{xurl.sty}{\usepackage{xurl}}{} % add URL line breaks if available
\urlstyle{same}
\hypersetup{
  pdftitle={p8105-hw1-yz4994},
  pdfauthor={YiwenZhang},
  hidelinks,
  pdfcreator={LaTeX via pandoc}}

\title{p8105-hw1-yz4994}
\author{YiwenZhang}
\date{2025-09-19}

\begin{document}
\maketitle

\section{Problem 1}\label{problem-1}

I start by loading the \texttt{moderndive} package and accessing the
\texttt{early\_january\_weather} dataset.I have installed the package
``moderndive'' in the console.

Now I load required libraries

\begin{Shaded}
\begin{Highlighting}[]
\FunctionTok{library}\NormalTok{(moderndive)}
\FunctionTok{library}\NormalTok{(tidyverse)}
\end{Highlighting}
\end{Shaded}

\begin{verbatim}
## -- Attaching core tidyverse packages ------------------------ tidyverse 2.0.0 --
## v dplyr     1.1.4     v readr     2.1.5
## v forcats   1.0.0     v stringr   1.5.1
## v ggplot2   3.5.2     v tibble    3.3.0
## v lubridate 1.9.4     v tidyr     1.3.1
## v purrr     1.1.0     
## -- Conflicts ------------------------------------------ tidyverse_conflicts() --
## x dplyr::filter() masks stats::filter()
## x dplyr::lag()    masks stats::lag()
## i Use the conflicted package (<http://conflicted.r-lib.org/>) to force all conflicts to become errors
\end{verbatim}

Now I load the dataset we need

\begin{Shaded}
\begin{Highlighting}[]
\FunctionTok{data}\NormalTok{(}\StringTok{"early\_january\_weather"}\NormalTok{)}
\FunctionTok{nrow}\NormalTok{(early\_january\_weather)}
\end{Highlighting}
\end{Shaded}

\begin{verbatim}
## [1] 358
\end{verbatim}

\begin{Shaded}
\begin{Highlighting}[]
\FunctionTok{ncol}\NormalTok{(early\_january\_weather)}
\end{Highlighting}
\end{Shaded}

\begin{verbatim}
## [1] 15
\end{verbatim}

\begin{Shaded}
\begin{Highlighting}[]
\FunctionTok{mean}\NormalTok{(early\_january\_weather}\SpecialCharTok{$}\NormalTok{temp, }\AttributeTok{na.rm =} \ConstantTok{TRUE}\NormalTok{)}
\end{Highlighting}
\end{Shaded}

\begin{verbatim}
## [1] 39.58212
\end{verbatim}

\begin{Shaded}
\begin{Highlighting}[]
\FunctionTok{glimpse}\NormalTok{(early\_january\_weather)}
\end{Highlighting}
\end{Shaded}

\begin{verbatim}
## Rows: 358
## Columns: 15
## $ origin     <chr> "EWR", "EWR", "EWR", "EWR", "EWR", "EWR", "EWR", "EWR", "EW~
## $ year       <int> 2013, 2013, 2013, 2013, 2013, 2013, 2013, 2013, 2013, 2013,~
## $ month      <int> 1, 1, 1, 1, 1, 1, 1, 1, 1, 1, 1, 1, 1, 1, 1, 1, 1, 1, 1, 1,~
## $ day        <int> 1, 1, 1, 1, 1, 1, 1, 1, 1, 1, 1, 1, 1, 1, 1, 1, 1, 1, 1, 1,~
## $ hour       <int> 1, 2, 3, 4, 5, 6, 7, 8, 9, 10, 11, 13, 14, 15, 16, 17, 18, ~
## $ temp       <dbl> 39.02, 39.02, 39.02, 39.92, 39.02, 37.94, 39.02, 39.92, 39.~
## $ dewp       <dbl> 26.06, 26.96, 28.04, 28.04, 28.04, 28.04, 28.04, 28.04, 28.~
## $ humid      <dbl> 59.37, 61.63, 64.43, 62.21, 64.43, 67.21, 64.43, 62.21, 62.~
## $ wind_dir   <dbl> 270, 250, 240, 250, 260, 240, 240, 250, 260, 260, 260, 330,~
## $ wind_speed <dbl> 10.35702, 8.05546, 11.50780, 12.65858, 12.65858, 11.50780, ~
## $ wind_gust  <dbl> NA, NA, NA, NA, NA, NA, NA, NA, NA, NA, NA, NA, NA, NA, 20.~
## $ precip     <dbl> 0, 0, 0, 0, 0, 0, 0, 0, 0, 0, 0, 0, 0, 0, 0, 0, 0, 0, 0, 0,~
## $ pressure   <dbl> 1012.0, 1012.3, 1012.5, 1012.2, 1011.9, 1012.4, 1012.2, 101~
## $ visib      <dbl> 10, 10, 10, 10, 10, 10, 10, 10, 10, 10, 10, 10, 10, 10, 10,~
## $ time_hour  <dttm> 2013-01-01 01:00:00, 2013-01-01 02:00:00, 2013-01-01 03:00~
\end{verbatim}

According to the help file of the dataset, I get the following
information:

1.Description:Hourly meteorological data for LGA, JFK and EWR for the
month of January 2013.The dataset is a data frame of 358rows
representing hourly measurements and 15 variables (358rows and 15
columns).

2.The mean temperature is 39.58212 F.

3.The variables in this dataset, including names / values of important
variables:

origin: Weather station, e.g.EWR

year:year of recording time, e.g.2013

month: month of recording time, e.g.1

day: day of recording time, e.g.1

hour: hour of recording time, e.g.3

temp: temperature in F, e.g.39.02

dewp: dewpoint in F, e.g.26.96

humid: relative humidity, e.g.59.37

wind\_dir: winf direction (in degree), e.g.270

wind\_speed: wind speed, e.g.10.35702

wind\_gust: wind gust speed (in mph), e.g.20

precip: precipitation, in inches, e.g.0

pressure: sea level pressure in millibars, e.g.1012.3

visib: visibility in miles, e.g.10

time\_hour: date and hour of the recording as a POSIXct date,
e.g.2013-01-01 01:00:00

\begin{Shaded}
\begin{Highlighting}[]
\FunctionTok{ggplot}\NormalTok{(early\_january\_weather, }\FunctionTok{aes}\NormalTok{(}\AttributeTok{x =}\NormalTok{ time\_hour, }\AttributeTok{y =}\NormalTok{ temp, }\AttributeTok{color =}\NormalTok{ humid)) }\SpecialCharTok{+}
  \FunctionTok{geom\_point}\NormalTok{(}\AttributeTok{alpha =} \FloatTok{0.6}\NormalTok{) }\SpecialCharTok{+}
  \FunctionTok{labs}\NormalTok{(}
    \AttributeTok{title =} \StringTok{"Temperature vs Time in Early January 2013"}\NormalTok{,}
    \AttributeTok{x =} \StringTok{"Time (hourly)"}\NormalTok{,}
    \AttributeTok{y =} \StringTok{"Temperature (°F)"}\NormalTok{,}
    \AttributeTok{color =} \StringTok{"Humidity"}
\NormalTok{  ) }\SpecialCharTok{+}
  \FunctionTok{theme\_minimal}\NormalTok{()}
\end{Highlighting}
\end{Shaded}

\pandocbounded{\includegraphics[keepaspectratio]{p8105-hw1-yz4994_files/figure-latex/unnamed-chunk-3-1.pdf}}

\subsection{Export the plot}\label{export-the-plot}

\begin{Shaded}
\begin{Highlighting}[]
\FunctionTok{ggsave}\NormalTok{(}\StringTok{"scatter\_temp\_time.png"}\NormalTok{, }\AttributeTok{width =} \DecValTok{7}\NormalTok{, }\AttributeTok{height =} \DecValTok{4}\NormalTok{)}
\end{Highlighting}
\end{Shaded}

\section{Problem 2}\label{problem-2}

This problem emphasizes variable types and coercion in R.

\subsection{1.Create the data frame}\label{create-the-data-frame}

\begin{Shaded}
\begin{Highlighting}[]
\FunctionTok{set.seed}\NormalTok{(}\DecValTok{123}\NormalTok{)   }\CommentTok{\# for reproducibility}

\NormalTok{df2 }\OtherTok{\textless{}{-}} \FunctionTok{tibble}\NormalTok{(}
  \AttributeTok{sample\_norm =} \FunctionTok{rnorm}\NormalTok{(}\DecValTok{10}\NormalTok{),                       }\CommentTok{\# numeric}
  \AttributeTok{logical\_vec =} \FunctionTok{rnorm}\NormalTok{(}\DecValTok{10}\NormalTok{) }\SpecialCharTok{\textgreater{}} \DecValTok{0}\NormalTok{,                   }\CommentTok{\# logical}
  \AttributeTok{char\_vec =} \FunctionTok{sample}\NormalTok{(letters, }\DecValTok{10}\NormalTok{, }\AttributeTok{replace =} \ConstantTok{TRUE}\NormalTok{),}\CommentTok{\# character}
  \AttributeTok{factor\_vec =} \FunctionTok{factor}\NormalTok{(}\FunctionTok{sample}\NormalTok{(}\FunctionTok{c}\NormalTok{(}\StringTok{"low"}\NormalTok{, }\StringTok{"med"}\NormalTok{, }\StringTok{"high"}\NormalTok{), }\DecValTok{10}\NormalTok{, }\AttributeTok{replace =} \ConstantTok{TRUE}\NormalTok{))}
\NormalTok{)}

\NormalTok{df2}
\end{Highlighting}
\end{Shaded}

\begin{verbatim}
## # A tibble: 10 x 4
##    sample_norm logical_vec char_vec factor_vec
##          <dbl> <lgl>       <chr>    <fct>     
##  1     -0.560  TRUE        o        low       
##  2     -0.230  TRUE        j        high      
##  3      1.56   TRUE        m        med       
##  4      0.0705 TRUE        g        low       
##  5      0.129  FALSE       i        med       
##  6      1.72   TRUE        i        low       
##  7      0.461  TRUE        j        low       
##  8     -1.27   FALSE       w        high      
##  9     -0.687  TRUE        u        low       
## 10     -0.446  FALSE       g        med
\end{verbatim}

\subsection{2. Take the mean of each
variable}\label{take-the-mean-of-each-variable}

\begin{Shaded}
\begin{Highlighting}[]
\FunctionTok{mean}\NormalTok{(}\FunctionTok{pull}\NormalTok{(df2,sample\_norm))   }\CommentTok{\# works: numeric}
\end{Highlighting}
\end{Shaded}

\begin{verbatim}
## [1] 0.07462564
\end{verbatim}

\begin{Shaded}
\begin{Highlighting}[]
\FunctionTok{mean}\NormalTok{(}\FunctionTok{pull}\NormalTok{(df2,logical\_vec))  }\CommentTok{\# works: TRUE = 1, FALSE = 0}
\end{Highlighting}
\end{Shaded}

\begin{verbatim}
## [1] 0.7
\end{verbatim}

\begin{Shaded}
\begin{Highlighting}[]
\FunctionTok{mean}\NormalTok{(}\FunctionTok{pull}\NormalTok{(df2,char\_vec))      }\CommentTok{\# fails: characters cannot be averaged}
\end{Highlighting}
\end{Shaded}

\begin{verbatim}
## Warning in mean.default(pull(df2, char_vec)): argument is not numeric or
## logical: returning NA
\end{verbatim}

\begin{verbatim}
## [1] NA
\end{verbatim}

\begin{Shaded}
\begin{Highlighting}[]
\FunctionTok{mean}\NormalTok{(}\FunctionTok{pull}\NormalTok{(df2,factor\_vec))    }\CommentTok{\# fails: factors not numeric by default}
\end{Highlighting}
\end{Shaded}

\begin{verbatim}
## Warning in mean.default(pull(df2, factor_vec)): argument is not numeric or
## logical: returning NA
\end{verbatim}

\begin{verbatim}
## [1] NA
\end{verbatim}

\subsection{3. Explanation}\label{explanation}

\begin{Shaded}
\begin{Highlighting}[]
\SpecialStringTok{{-} }\NormalTok{**Numeric variable**: Taking the mean works as expected.  }
\SpecialStringTok{{-} }\NormalTok{**Logical variable**: Taking the mean works; R treats }\InformationTok{\textasciigrave{}TRUE = 1\textasciigrave{}}\NormalTok{ and }\InformationTok{\textasciigrave{}FALSE = 0\textasciigrave{}}\NormalTok{. The mean is the proportion of TRUE values.  }
\SpecialStringTok{{-} }\NormalTok{**Character variable**: Taking the mean fails because characters cannot be coerced to numbers automatically.  }
\SpecialStringTok{{-} }\NormalTok{**Factor variable**: Taking the mean fails because factors are categorical labels, not numeric.  }
\end{Highlighting}
\end{Shaded}

\subsection{\texorpdfstring{4. Using
\texttt{as.numeric}}{4. Using as.numeric}}\label{using-as.numeric}

\begin{Shaded}
\begin{Highlighting}[]
\FunctionTok{as.numeric}\NormalTok{(}\FunctionTok{pull}\NormalTok{(df2,logical\_vec))   }\CommentTok{\# converts TRUE/FALSE → 1/0}
\FunctionTok{as.numeric}\NormalTok{(}\FunctionTok{pull}\NormalTok{(df2,char\_vec))      }\CommentTok{\# returns NA with warning (cannot convert letters to numbers)}
\end{Highlighting}
\end{Shaded}

\begin{verbatim}
## Warning: NAs introduced by coercion
\end{verbatim}

\begin{Shaded}
\begin{Highlighting}[]
\FunctionTok{as.numeric}\NormalTok{(}\FunctionTok{pull}\NormalTok{(df2,factor\_vec))    }\CommentTok{\# converts factor levels → underlying integer codes (1, 2, 3)}
\end{Highlighting}
\end{Shaded}

\subsection{Explanation}\label{explanation-1}

\begin{Shaded}
\begin{Highlighting}[]
\SpecialStringTok{{-} }\NormalTok{Converting **logical** to numeric turns TRUE/FALSE into 1/0, which explains why taking the mean of a logical vector gives the proportion of TRUEs.  }
\SpecialStringTok{{-} }\NormalTok{Converting **character** to numeric produces NAs (with a warning), since characters like "a", "b", "c" have no numeric representation.  }
\SpecialStringTok{{-} }\NormalTok{Converting **factor** to numeric produces the underlying integer codes of the factor levels, not the labels. This explains why I can’t take a mean of factors directly — the numbers would represent codes, not meaningful values.  }
\end{Highlighting}
\end{Shaded}


\end{document}
